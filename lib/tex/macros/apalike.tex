%% @texfile{
%%   author = "Oren Patashnik",
%%   version = "0.99a",
%%   date = "12 Dec 1990",
%%   filename = "apalike.tex",
%%   address = "Please use electronic mail",
%%   checksum = "88     707    4412",
%%   email = "opbibtex@neon.stanford.edu",
%%   codetable = "ISO/ASCII",
%%   supported = "yes",
%%   docstring = "Defines macros that make apalike work with plain TeX",
%% }
% apalike.tex, version 0.99a, for btxmac 0.99f, BibTeX 0.99c, TeX 3.0 or later.
% Copyright (C) 1990; all rights reserved.
% You may copy this file provided: that it's accompanied by btxmac.tex;
% and that either you make absolutely no changes to your copy, or if you
% do make changes, (1) you name the file something other than
% `apalike.tex' and you remove all occurrences of `apalike.tex' from the
% file, (2) you put, somewhere in the first twenty lines of the file,
% your name, along with an electronic address at which others who might
% use the file may reach you, and (3) you remove each occurrence of
% Oren's name and electronic address from this file.  These restrictions
% help ensure that all standard versions of these macros are identical,
% and that Oren doesn't get deluged with inappropriate e-mail.
%
% This file, apalike.tex, contains TeX macros that let you use the
% apalike bibliography style with plain TeX.  In essence, this file
% provides the Tex counterpart to apalike.sty, the LaTeX style file
% required for using the apalike bibliography style.  Please report any
% bugs (outright goofs, misfeatures, or unclear documentation) to Oren
% Patashnik (opbibtex@neon.stanford.edu).  These macros will become
% frozen shortly after BibTeX version 1.00 is released.
%
% Editorial note (i.e., flame):
% Many journals require a style like `apalike', but I strongly, strongly,
% strongly recommend that you not use it if you have a choice---use
% something like `plain' instead.  Mary-Claire van Leunen (A Handbook for
% Scholars, Knopf, 1979) argues convincingly that a style like `plain'
% encourages better writing than one like `apalike'.  Furthermore the best
% argument for using an author-date style like `apalike'---that it's "the
% most practical" (The Chicago Manual of Style, University of Chicago
% Press, thirteenth edition, 1982, pages 400--401)---falls flat on its
% face with the new computer-typesetting technology.  For instance page 401
% of the Chicago Manual anachronistically states "The chief disadvantage of
% [a style like `plain'] is that additions or deletions cannot be made
% after the manuscript is typed without changing numbers in both text
% references and list."  With LaTeX the disadvantage obviously evaporates.
% Moreover, apalike indulges in what I think is an execrable practice:
% automatically abbreviating first names.  It's true that there aren't very
% many D. E. Knuth's around, so abbreviating his first name doesn't cause
% much confusion.  But I've personally known three D. E. Smith's, two of
% whom have published in the same field; abbreviating their first names
% *is* confusing.  Especially with all the citation indexes nowadays, it's
% better to give first names exactly as they appear in the source being
% cited.  Automatically abbreviating first names is simply bad scholarship.
% (End of flame.)
%
% To use these macros you need the btxmac macros, whose purpose is to
% let you use BibTeX with plain TeX (rather than with LaTeX); the file
% btxmac.tex explains those macros in detail.  You simply \input apalike
% right after you \input btxmac to invoke these macros.
%
%
%   HISTORY
%
% Oren Patashnik wrote the original version of these macros in December
% 1990, for use with btxmac.tex.
%
%   12-Dec-90  Version 0.99a, first general release.
%
%
% Here, finally (I swear, I thought he was never gonna stop), are the
% macros.  The first bunch makes the label empty and sets 2em of
% hanging indentation for each entry.
%
\def\biblabelprint#1{\noindent}%
\def\biblabelcontents#1{}%
\def\bblhook{\biblabelextrahang = 2em}%
%
%
% And the last bunch formats an in-text citation: parens around the
% entire citation; semicolons separating individual references; and a
% comma between a reference and its note (like `page 41') if it exists.
%
\def\printcitestart{(}%         left paren
\def\printcitefinish{)}%        right parent
\def\printbetweencitations{; }% semicolon, space
\def\printcitenote#1{, #1}%     comma, space, note (if it exists)
