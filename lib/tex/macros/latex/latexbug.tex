% \iffalse meta-comment
%
% Copyright 1994 the LaTeX3 project and the individual authors.
% All rights reserved. For further copyright information see the file
% legal.txt, and any other copyright indicated in this file.
% 
% This file is part of the LaTeX2e system.
% ----------------------------------------
% 
%  This system is distributed in the hope that it will be useful,
%  but WITHOUT ANY WARRANTY; without even the implied warranty of
%  MERCHANTABILITY or FITNESS FOR A PARTICULAR PURPOSE.
% 
% 
% IMPORTANT NOTICE:
% 
% For error reports in case of UNCHANGED versions see bugs.txt.
% 
% Please do not request updates from us directly.  Distribution is
% done through Mail-Servers and TeX organizations.
% 
% You are not allowed to change this file.
% 
% You are allowed to distribute this file under the condition that
% it is distributed together with all files mentioned in manifest.txt.
% 
% If you receive only some of these files from someone, complain!
% 
% You are NOT ALLOWED to distribute this file alone.  You are NOT
% ALLOWED to take money for the distribution or use of either this
% file or a changed version, except for a nominal charge for copying
% etc.
% \fi

%%% ====================================================================
%%%  @LaTeX-file{
%%%     author          = "David Carlisle",
%%%     version         = "1.08",
%%%     date            = "08 June 1994",
%%%     filename        = "latexbug.tex",
%%%     email           = "latex-bugs@uni-mainz.de",
%%%     codetable       = "ISO/ASCII",
%%%     keywords        = "LaTeX, bugs, reporting",
%%%     supported       = "yes",
%%%     docstring       = "
%%%
%%%     LaTeX bug report generator.
%%%     %%%%%%%%%%%%%%%%%%%%%%%%%%
%%%
%%%     Processing this file with LaTeX should produce
%%%     a file latexbug.msg in the current directory.
%%%
%%%     latexbug.msg may be used as a template for submitting bug
%%%     reports concerning files in the standard LaTeX distribution.
%%%
%%%     Any bug report should include a small LaTeX test file
%%%     that shows the bug, and the log that LaTeX produces on the
%%%     test file.
%%%
%%%     Please check before submitting a bug report that your format is
%%%     not more than one year old. New LaTeX releases occur at 6
%%%     monthly intervals, and so your bug may be fixed in a later
%%%     release.
%%%
%%%     Completed bug report forms should be submitted to:
%%%     latex-bugs@uni-mainz.de
%%%
%%%  }
%%% ====================================================================

\ifcat{=

\catcode`\{=1 \let\bgroup{
\catcode`\}=2 \let\egroup}
\catcode`\#=6 
\catcode`\^=7
\catcode`\@=11

\newlinechar`\^^J
\def\m@ne{-1 }
\countdef\count@255

\def\fmtname{INITEX}
\def\fmtversion{9999/00/00}
\def\space{ }
\def\@spaces{\space\space\space\space}
\let\@@end\end

\chardef\msg15
\immediate\openout\msg=\jobname.msg

\expandafter\edef\csname newif\endcsname#1#2{%
  \let\noexpand\ifinteractive
    \expandafter\noexpand\csname iffalse\endcsname}

\def\typeout{\immediate\write17}

\def\two@digits#1{\ifnum#1<10 0\fi\number#1}

\def\wmsg#1#{\bgroup\@wmsg}

\else
%%
%% @ is a letter
%%
\catcode`\@=11

%%
%% Output stream to produce the bug report template.
%%
\newwrite\msg
\immediate\openout\msg=\jobname.msg


%%
%% Check that LaTeX2e is being used.
%%
\ifx\undefined\newcommand
 \newlinechar`\^^J%
 \immediate\write17{^^J%
    You must use LaTeX2e to generate the bug report!^^J^^J%
    If there is a bug in the installation procedure,^^J%
    and you can not create LaTeX2e, you may use initex^^J%
    to generate the report}%

 \let\relax\end
\else
\def\@tempa{LaTeX2e}\ifx\@tempa\fmtname\else
 \immediate\write17{^^J%
  Older Versions of LaTeX are no longer supported.^^J%
  You must use LaTeX2e to generate the bug report!^^J^^J%
  If there is a bug in the installation procedure,^^J%
  and you can not create LaTeX2e, you may use initex^^J%
  to generate the report}%
 \let\relax\@@end
\fi\fi

%%
%% \wmsg writes to the terminal, and the .msg file
%% \wmsg* just writes to the .msg file
%% \typeout just writes to the terminal
%%

\def\wmsg{\bgroup\@ifstar{\interactivefalse\@wmsg}\@wmsg}

\fi

\relax


\def\@wmsg#1{%
  \ifinteractive\immediate\write17{#1}\fi
  \immediate\write\msg{#1}%
  \egroup}


%%
%% if \interactivefalse just make a blank template.
%%
\newif\ifinteractive
\interactivetrue


%%%
%%% For testing for blank lines.
%%%
\def\xpar{\par}


%%
%% Opening Banner.
%%
\typeout{^^J%
============================================================^^J%
^^J%
LaTeX bug report generator^^J%
==========================^^J%
Processing this file with LaTeX will produce a template \jobname.msg^^J%
for submitting bug reports for the LaTeX distribution.^^J%
Please do not report bugs in other, non-standard, files to the^^J%
latex-bugs address.^^J}

\count@=0
\ifinteractive
\typeout{%
Several categories of files are supported,^^J%
corresponding to directories in the standard LaTeX distribution:^^J^^J
0) base: \@spaces
         The format itself, and the main classes such as `article'.^^J
1) tools:\@spaces
         Packages supported by the LaTeX3 project team.^^J
2) graphics: 
         The color and graphics packages.^^J
3) mfnfss: \space\space
         Packages for using MetaFont fonts with NFSS (ie LaTeX2e).^^J
4) psnfss: \space\space
         Packages for using PostScript fonts with NFSS (ie LaTeX2e).^^J
5) amslatex:
         Classes and Packages supported by the AMS.^^J
6) babel:\@spaces
         Packages supporting many different languages.^^J%
}
\message{Please select a category 0--6:  }
\read\m@ne to \answer
\count@=\answer
\else
\typeout{As you are using INITEX, I will assume category 0}
\fi

\ifcase\count@
\wmsg{^^J>Category: latex}\or
\wmsg{^^J>Category: tools}\or
\wmsg{^^J>Category: graphics}\or
\wmsg{^^J>Category: mfnfss}\or
\wmsg{^^J>Category: psnfss}\or
\wmsg{^^J>Category: amslatex}\or
\wmsg{^^J>Category: babel}\else
\errhelp{Quit with `x' and then re-start latexbug}
\def\badcategory{Only categories 0,...,6 are supported at this time}
\errmessage{\badcategory}
\fi


\typeout{%
^^J%
\ifinteractive
This report generator may be used in one of two ways.^^J%
If you choose the interactive option, you will be prompted to answer^^J%
several questions. Otherwise a blank template will be created for^^J%
you to fill in using your editor.^^J%
\else
INITEX should only be used for reporting bugs with the LaTeX2e^^J%
installation procedure. If you have a working copy of LaTeX2e,^^J%
please use that to generate the report.
\fi}

\ifinteractive
\message{Interactive session (y/n) ? }
\read\m@ne to \answer

\ifx\answer\xpar
   \def\answer{n}
\fi

%%
%% Allow anything begining with `y' or `Y' for yes.
%%
\edef\answer{\uccode`\expandafter\@car\answer\@nil}
\ifnum \answer=`Y \else\interactivefalse\fi
\else
\def\answer{`\N}
\fi

%%
%% Header in the msg file.
%%
\wmsg*{^^J%
 LaTeX2e bug report.^^J%
\ifnum \answer=`Y Generated \else Template generated \fi
 by latexbug.tex on \number\year/\two@digits\month/\two@digits\day^^J%
 ============================================================}

\ifinteractive
%%
%% if interactive, \wread reads a line (verbatim) and write it to the
%% .msg file, until a blank line is entered.
%%
  \typeout{^^J%
    The answer to each question may take several lines.^^J%
    (Each line will be prompted by =>.)^^J%
    Typing a blank line terminates the current answer.}
  \def\wread{{\catcode`\^^I=12
  \let\do\@makeother\dospecials
  \message{=> }\read\m@ne to \answer
  \ifx\answer\xpar\else
    \immediate\write\msg{\answer}
    \expandafter\wread
  \fi}}
\else
%%
%% If non-interactive, \wread just writes a blank line to the .msg file,
%% and \wmsg does not write to the terminal.
%%
  \def\wread{\wmsg{}}
\fi

%%
%% \copytomsg copies the contents of a file into the .msg file.
%% (at least it does it as well as TeX can, so there may be
%% transcription problems with 8-bit characters).
%%

\chardef\inputfile=15

\def\copytomsg#1{{%
   \def\do##1{\catcode`##1=11}%
   \dospecials
   \openin\inputfile#1
   \@copytomsg
   \closein\inputfile
}}

\def\@copytomsg{%
   \ifeof\inputfile
   \else
      \read\inputfile to \inputline
      \ifx\inputline\xpar
         \wmsg*{}
      \else
         \wmsg*{\inputline}
      \fi
      \expandafter\@copytomsg
   \fi
}

%% 
%% \stripspaces...\endstripspaces strips space off the end of its
%% argument. 
%%

\def\stripspaces#1 \endstripspaces{#1}

%%
%% Test which format is being used. These fields are completed
%% automatically even if the blank template is being produced.
%%
\wmsg{^^J%
Format:^^J%
\expandafter\@secondoftwo\the\everyjob}

%%
%% Test the age of the current format.
%%
\def\getage#1/#2/#3\@nil{%
  \count@\year
  \advance\count@-#1\relax
  \multiply\count@ by 12\relax
  \advance\count@\month
  \advance\count@-#2\relax}
%
\expandafter\getage\fmtversion\@nil
%%
%% \count@ should now be the age of the format in months.
%%
\ifnum\count@>12
\def\oldformat{^^J%
   ! Your LaTeX installation is more than one year old.^^J%
   ! Please consider updating LaTeX before submitting this report.^^J%
   ! At least check a current latex.bug file, to see if the bug^^J%
   ! has been fixed in the current release.^^J%
   !}
%%
%% Put the message in a macro to improve the look of the error mesage.
%%

\errhelp{If you still wish to complete the form, just type return.}
\errmessage{\oldformat}
\fi

%%
%% Three macros that vary their definitions betwen LaTeX installations.
%%
\wmsg*{^^J%
TeX Version:^^J%
\meaning\@TeXversion}

\wmsg*{^^J%
Current Directory Syntax:^^J%
\meaning\@currdir}

\wmsg*{^^J%
Input Path:^^J%
\meaning\input@path}


%%
%% Now just use \wmsg and \wread for each of the fields in the form.
%%
\wmsg{^^J%
Your Name:}
\wread

\wmsg{^^J%
Your Address (preferably email):}
\wread

\wmsg{^^J%
Brief description of bug:}
\wread

\ifinteractive
   \wmsg{^^J%
      Name of a SHORT, SELF-CONTAINED file which indicates the problem:}
   \message{=> }\read\m@ne to \filename
\else
   \def\filename{\par}
\fi

%% 
%% Try to find the .tex file and .log file
%%

\ifx\filename\xpar

\ifx\LaTeX\undefinedcommand\else

\wmsg*{^^J%
Sample file which indicates the problem:^^J%
========================================^^J%
^^J%
The log file from running LaTeX on the sample:^^J%
==============================================}

\fi

\else

\wmsg*{\filename}

\edef\filename{\expandafter\stripspaces\filename\endstripspaces}

\filename@parse\filename

\IfFileExists{\filename}{
   \edef\samplefile{\filename}
}{}

\IfFileExists{\filename@area\filename@base.log}{
   \edef\logfile{\filename@area\filename@base.log}
}{
   \IfFileExists{\filename@area\filename@base.lis}{
      \edef\logfile{\filename@area\filename@base.lis}
   }{}
}


%% 
%% The example file goes here:
%%

\wmsg*{^^J%
Sample file which indicates the problem:^^J%
========================================}

\ifx\samplefile\undefinedcommand
   \typeout{^^J%
      Sample file \filename not found.^^J%
      Please edit \jobname.msg to include the sample file.}
\else
   \copytomsg{\samplefile}
\fi


%%
%% The log file goes here:
%%
\wmsg*{^^J%
The log file from running LaTeX on the sample:^^J%
==============================================}

\ifx\logfile\undefinedcommand
   \typeout{^^J%
      Log file \filename@area\filename@base.log not found.^^J%
      Please edit \jobname.msg to include the log file.}
\else
   \copytomsg{\logfile}
\fi

\fi


%%
%% Closing Banner.
%%
\typeout{^^J%
============================================================}

\ifinteractive
 \typeout{^^J%
   You may wish to make further changes to the bug report file:^^J%
   `\jobname.msg'^^J%
   using your editor.}
\else
 \typeout{^^J%
   A template for submitting bug reports has been left in the file:^^J%
   \jobname.msg^^J%
   Please use your editor to complete the file before submitting^^J%
   your report.}
\fi

\let\ifinteractivetrue\iftrue
\typeout{^^J%
============================================================^^J%
^^J%
  If you have access to email, please send `\jobname.msg' to:^^J%
  latex-bugs@uni-mainz.de^^J%
^^J%
  Thank you for taking the time to submit a bug report.^^J
^^J%
============================================================}

\wmsg*{^^J%
============================================================^^J
^^J%
   End of LaTeX2e bug report.^^J%
^^J%
============================================================}

%%
%% Close the .msg output stream.
%%
\immediate\closeout\msg

%%
%% This is the TeX primitive \end command.
%%
\@@end
