\documentstyle{tm}
\newcommand{\BibTeX}{{\rm B\kern-.05em{\sc i\kern-.025em b}\kern-.08em
    T\kern-.1667em\lower.7ex\hbox{E}\kern-.125emX}}
\markproprietary
\title{AT\&T Specific \LaTeX\ Document Styles and Related Tools}
\author{Terry L Anderson}
\signatureextra{Software Development Environment Group\\Software Architecture Department}%
\initials{TLA}
\eaddress{tla@bartok.att.com}
\department{59114}
\location{LC}{4N-E01}{908-580-4428}
\date{Nov 14, 1991}
\chargecase{978899-0100}
\filecase{61093}
\documentno{910000}{00}{TMS}
\previousdocument{59553-870625-02TM, 59114-890130-01TMS}
\keywords{\TeX ; \LaTeX ; Memorandum; View Graphs; Slides; voucher;
mm; tm.sty;
Document Preparation}
\mercurycode{CMP}
\memotype{TECHNICAL MEMORANDUM}
\organizationalapproval
\approver{T.\ F.\ Houghton, Supv}
\approver{G.\ P.\ Pasternack, Head 59112}
\abstract{
This document describes several AT\&T specific document styles for
\LaTeX , including {\it tm} style for preparing AT\&T Bell
Laboratories style memoranda using \LaTeX , a companion style, {\it
rp}, for released papers, a letter style, {\it attletter} that
includes an AT\&T letterhead, {\it mw.sty} for preparing
documents following the Multiweight Design format (an AT\&T standard
for external, end-user documentation), a viewgraph style,
{\it attslides}, and a style for preparing Bell Laboratories expense
vouchers, {\it voucher.sty}.  Also described in a translator from
mm/troff to \LaTeX .Most of the features of
the mm macro package available under troff are available in \LaTeX ,
except for the built-in knowledge of AT\&T memorandum layout styles
and coversheets.  The {\it tm} document style adds support for all of
the special memorandum information, such as work locations, document
numbers, work project numbers, MERCURY codes, \dots , and will produce
coversheets, contents lists, memorandum headers, proprietary markings,
\dots\ in a fairly standard form.  A related {\it rp} style for
released papers accepts the same document file but produces a paper
approximating mm's released paper memorandum type.  A letter style,
{\it attletter}, similar to \LaTeX 's standard letter style but adding
the AT\&T letterhead is also described.  Examples of technical
memorandums and letters are included.  The style files and other
related files are available on request.  Some extended picture
environment macros, macros for printing bitmaps and the AT\&T Logo
defined in METAFONT are also available.}
\begin{document}
\makehead
\makeabstract
\section{Introduction}
This document describes a number of AT\&T specific \LaTeX\ styles and
some related tools.  The AT\&T memo style {\it tm} is described in
detail.  The {\it voucher} style for preparing Bell Laboratories
vouchers is also documented.  A special style for SLI\TeX\, {\it
attslides} for preparing one of the many AT\&T slide styles is also
documented.  A style for preparing documents in Multiweight Design, an
AT\&T standard for external, end-user documentation is descibed here
but documented fulling in another document.  See {\it A \LaTeX\ Style for
Multiweight Design}.\cite{bib:mw}  The mm/troff translator is
documented in {\it An MM to \LaTeX\ Translator}.\cite{bib:mm2tex}

\part{Memo Styles, TM and RP}
\LaTeX\ is a macro package for use with the typesetting language
\TeX\tmnotice{\TeX\ is a trademark of the American
Mathematical Society}.  It simplifies a number of formatting 
tasks common in producing large documents.  
\LaTeX\ supports document style descriptions that specify
the layout of information in a document and on each page.  This
document describes a document style that produces 
AT\&T Bell Laboratories memoranda, in a form similar to those produced
with the mm macro package\cite{bib:mm,bib:mm2} available under {\it troff}.  
It is assumed that the reader is reasonably familiar with \LaTeX ,
and its use.  This document will not discuss the standard features of \LaTeX . 
The {\it tm} style is a standard \LaTeX\ document style file and is
very similar to the {\it article} style described in the \LaTeX\
book.\cite{bib:latex}  It is invoked by the command
\begin{itemize}
\item[]
\verb|\documentstyle{tm}|
\end{itemize}
at the beginning of the document source file.  We will document only 
the differences and extensions to the {\it article}
style.  This memorandum documents the version of {\it tm.sty} dated
87-Jul-10.\markedfootnote{\dag}{The version date is printed on the
terminal when the style file is read into \TeX\ and at the bottom of
the first page of the coversheet.}  The {\it tm} style was written
by the author.

\section{Differences from {\it article} Style Defaults}
The text width and height dimensions have been changed to correspond
to those used by mm memoranda; text width is 6 inches and height
8.7.  A file, named ``local.sty'', is read in to allow local page
positioning parameters.  This is for use with dvi programs that cannot
change the offsets just before printing.  This file must exist but
may be empty.

Several other changes in format have been made to conform more
closely to standard memorandum style.  The format of the table of
contents has been adjusted.  Calls to \verb|\newpage| have been
added before sections such as references and table of contents to
make them occur on new pages.

The \verb|\cite| macro has been changed to print square brackets
around the entry and to raise it above the
line.\markedfootnote{*}{See an example in the first paragraph 
of this document}  The citation marking can be left unraised by
calling 
\begin{itemize}
\item[]
\verb|\noraisedcitations|
\end{itemize}
and raised again by calling
\begin{itemize}
\item[]
\verb|\raisedcitations|
\end{itemize}
The default is raised.

The {\it tm} style supports most of the options of the article style,
such as different default type sizes [11pt] or [12pt] but the two
column option is not recommended.  {\it tm} also adds support for some
smaller point sizes, [8pt] and [9pt].
\section{Memorandum Data}
The {\it tm} style adds commands for giving values to various
memorandum parameters so that they can be printed in standard form
on the first page heading, in the end of document signature and on
coversheets.
Several of these replace similar commands in the {\it article}
style; in a few cases with different parameters.  

The title can be entered using
\begin{itemize}
\item[]
\verb|\title{|{\it text}\verb|}|\\
\verb|\title*{|{\it text}\verb|}|
\end{itemize}
When using the default page style, the title is automatically placed
in the appropriate places in the first page document heading, the
coversheet (if generated) and at the top of each page in the page
heading (see section on New Pages Styles for other options).  The
title at the top of each page can be suppressed by using
\verb|\title*|.  The
macro \verb|\title| actually sets three internal macros for printing
the title on the title page, on the coversheet and on the top of each
page.  If the title is too long to fit comfortably in the heading, a
shortened version, some other appropriate heading or no heading at all
can be substituted by using
\verb|\markright{|{\it string}\verb|}| or 
\verb|\markright{}|.\markedfootnote{\ddag}{See the description of
the {\it headandfoot} page style below}  
If it is desired to specify line breaks for one of the other titles,
the individual macros 
\begin{itemize}
\item[]
\verb|\covtitle{|{\it text}\verb|}|\\
\verb|\titlepagetitle{|{\it text}\verb|}|
\end{itemize}
can be called after calling \verb|\title|.  These macros will allow
`\verb|\\|' to appear in the title.  If \verb|\title| is not called
all three macros \verb|\markright|, \verb|\covtitle| and
\verb|titlepagetitle| must be called.

The author and associated data is entered as follows.
\begin{itemize}
\item[]
\verb|\author{|{\it author\/}\verb|}|\\
\verb|\initials{|{\it initials\/}\verb|}|
\end{itemize}
These are traditionally the author's three initials in upper case,
and will appear along with location code and department number in
the signature block.  The command
\begin{itemize}
\item[]
\verb|\signatureextra{|{\it string}\verb|}|
\end{itemize}
allows any additional information, such as the authors
title, department name, \dots , to be added directly under the
author's name (as with the .AT macro
in mm).
\begin{itemize}
\item[]
\verb|\eaddress{|{\it electronic-address\/}\verb|}|
\end{itemize}
The electronic address (if given) will be displayed below the office
number and phone extension in first page header and in a column
after the name on the ITDS-style coversheet.
\begin{itemize}
\item[]
\verb|\department{|{\it department-number\/}\verb|}|\\
\verb|\location{|{\it location-code\/}\verb|}{|{\it
room-number\/}\verb|}{|{\it phone-number\/}\verb|}|\\
\verb|\locationextra{|{\it string}\verb|}|
\end{itemize}
{\it Note that effective mid-1990 the AT\&T memorandum style is to use
complete phone numbers including area code, not just extensions} -- the
argument to \verb|location| should thus contain the full phone number
in the form \verb|###-###-####|.  Effective with the 90-Sep-11 release
of tm.sty, the author information in the `from' part will use the new
layout with a separate line for phone number and no `x' in front of
the number.
The \verb|\locationextra| entry allow any additional information desired
to be included with the location information in the first page
header.  It appears under the the electronic address (if given).

A company name and mailing address can be given by using
\begin{itemize}
\item[]
\verb|\company{|{\it company-name\/}\verb|}|\\
\verb|\maddress{|{\it street-address\verb|\\|city, ST ZIP\/}\verb|}|
\end{itemize}
As many lines as necessary can be given in the mailing address as
long as they are separated by `\verb|\\|'.  The company name will
appear on the ITDS-style coversheet in the `Company\dots' field, and the
company name as well as any mailing address will appear on the first
page.  Since this is primarily intended for authors who are not 
AT\&T Bell Laboratories employees, the location,
department and document numbers would normally be left blank.
For employees of other AT\&T entities, the entity name can be given
as the company and location and office information specified as for
a Bell Laboratories employee.

All of the author related data macros can be collected into a author
database.  For more information see the section on {\it Author
Database}. 

The document number may be specified using
\begin{itemize}
\item[]
\verb|\documentno{|{\it yymmdd\/}\verb|}{|{\it
sequence-num\/}\verb|}{|{\it category-with-suffix\/}\verb|}|
\end{itemize}
Note that the document number does {\it
not} include the department number part.  This is added
automatically.  The {\it category-with-suffix} is `IM', `TM' or `TC'
with `S' added if the document includes ``any nontrivial executable
computer program routine, whether in source or object code (See GEI
2.2-2.)''.

The author related data macros \verb|\author|,
\verb|\initials|, \verb|\department|, and \verb|\location|
are all {\it required}.  Leaving any out will cause \TeX\ undefined
macro errors.  All of the data macros associated with the author
(the required macros plus \verb|\signatureextra|, \verb|\eaddress|, 
\verb|\locationextra|, and \verb|\documentnumber|) can be repeated 
for more authors.  Only the first nine authors names will have
signature lines on the ITDS-style coversheet.
The \verb|\author| should occur first and all other data up to the 
next \verb|\author| applies to that
author.  Declaring another author without supplying all the required
data macros will cause a \TeX\ error.  If the document number is
given for an author, it will appear under his {\it author data} in
the heading.  If all author's document numbers are the same it
should be specified only once with the last author.  Then it will
appear only at the end of the author list in the heading.  If there
is more than one document number, a document number should be given
for each author.  This will cause it appear after each author's data.  
In any case, each unique document number will occur only once on 
the cover sheet (if generated).

A typist's initials can be specified by
\begin{itemize}
\item[]
\verb|\typistinitials{|{\it initials\/}\verb|}|
\end{itemize}
The initials for the typist are usually given as lower-case letters.
Specifying a typist's initials is optional but if they are given they
will appear after the author's initials in the signature block.

The document numbers of previous documents that are amended or
superseded by the current document may also be specified by
\begin{itemize}
\item[]
\verb|\previousdocument{|{\it documentnumber\/}\verb|}|
\end{itemize}
The full document number (including department number, date,
sequence number and any suffix) should be given exactly as it should
appear.  Numbers in the older document number style can also be
used.  If more than one previous document number needs to be
specified, use one \verb|\previousdocument| macro and separate
the document numbers by commas.

The date is entered with
\begin{itemize}
\item[]
\verb|\date{|{\it date\/}\verb|}|
\end{itemize}
This is identical to the {\it date} macro in mm.  It is common to
use the \TeX\ macro \verb|\today| to generate the current date.  If
you do not call \verb|\date| the date will default to the current date.
\begin{itemize}
\item[]
\verb|\chargecase{|{\it charge-case-num\/}\verb|}|\\
\verb|\filecase{|{\it file-case-num\/}\verb|}|
\end{itemize}
These are fairly obvious.  Just as with the author data the charge
case and file case macros can be repeated.  They are not associated
with a particular author so they can appear anywhere in the
preamble, but they will be listed in the order given, in the heading
and on the coversheet.
\begin{itemize}
\item[]
\verb|\keywords{|{\it keyword-list\/}\verb|}|\\
\verb|\mercurycode{|{\it code\/}\verb|}|
\end{itemize}
The keyword list is simply a string, punctuated as desired.  It is
common to place `;' between phrases, as in ``document preparation;
typesetting; memorandum''.  The string will be printed exactly as
given.  The {\it mercury code} must be from
the standard list CHM, CMM, CMP, ELC, LFS, MAN, MAS, MKT, PHY, or
STD, and should be uppercase.  
The \verb|\mercurycode| macro can be repeated for more than
one code.  Only three mercury codes will appear on the ITDS-style
coversheet.
\begin{itemize}
\item[]
\verb|\memotype{|{\it type-name\/}\verb|}|
\end{itemize}
This string will appear in italics just under the header on the
first page.  Common {\it type-name}s are `TECHNICAL MEMORANDUM'
and `INTERNAL MEMORANDUM' but any string may be used.  This is the
same function as the `.MT ``string'' ' macro in mm.
\begin{itemize}
\item[]
\verb|\abstract{|{\it text}\verb|}|
\end{itemize}
Note that this is a macro, not an environment as in the standard {\it
article} style.  This is to allow the data to be captured and
printed in more than one place.  This also means that it should
appear in the preamble (along with the rest of the items in this
section) rather than in the body of the document or immediately after
the \verb|\begin{document}| (and before the \verb|\makeabstract|
described next).  If the abstract contains citations it {\it must}
follow the \verb|\begin{document}| to allow the citations to be
resolved.  The abstract 
can be placed at the top of the first page (just under
the memotype), by calling 
\begin{itemize}
\item[]
\verb|\makeabstract|
\end{itemize}
It will also be placed automatically on the coversheet (if one is
generated).
\begin{itemize}
\item[]
\verb|\copyto{|{\it name\/}\verb|\\|{\it name\/}\verb|\\|{\it name \dots}\verb|}|\\
\verb|\copytocov{|{\it name\/}\verb|\\|{\it name\/}\verb|\\|{\it name \dots}\verb|}|\\
\verb|\copytohere{|{\it name\/}\verb|\\|{\it name\/}\verb|\\|{\it name \dots}\verb|}|\\
\verb|\coverto{|{\it name\/}\verb|\\|{\it name\/}\verb|\\|{\it name \dots}\verb|}|
\end{itemize}
The \verb|\\| 
must be placed after each name except the last.

These lists will appear on the ITDS-style coversheet and the {\it copyto} list
will appear at the location where the macro is called (usually after
the signature.  \verb|\copytocov| suppresses the local copy, printing
the copyto list only on the coversheet.  \verb|\copytohere| suppresses
the copy on the coversheet, printing only at the location of the macro
call.  

Copyto lists that are longer than the space remaining on the signature
page are automatically continued onto the next page, but copyto and
coverto lists that are too long for the space available on the
coversheet are {\it not} automatically handled.  The lists must be
manually broken into to lists.  A list short enough to fit in the
space on the coversheet should be given with one of the above
commands.  The additional list can be created with one of the following

\begin{itemize}
\item[]
\verb|\restofcopyto{|{\it name\/}\verb|\\|{\it name\/}\verb|\\|{\it name \dots}\verb|}|\\
\verb|\restofcopytocov{|{\it name\/}\verb|\\|{\it name\/}\verb|\\|{\it name \dots}\verb|}|\\
\verb|\restofcoverto{|{\it name\/}\verb|\\|{\it name\/}\verb|\\|{\it name \dots}\verb|}|
\end{itemize}

There is no \verb|\restofcopytohere| since if the only copy is to be
on the signature page the standard \verb|\copytohere| will handle the
long list.  The \verb|\restofcopyto| will add its list immediately
after the standard copyto list on the signature page to look like a
single long list.  In addition, it along with the other two, will
generate an additional coversheet page and include the extra list or
lists.  This page will have the appropriate heading material.  No
provision is made for more than this single extra page.  It is also
possible to use a tabular environment in any of the lists to make
use of multiple columns.

A government security classification can be added by calling
\begin{itemize}
\item[]
\verb|\classified{|{\it classification-level}\verb|}|
\end{itemize}
The {\it classification-level} is the government assigned title;
e.g., Secret, Top Secret, \dots  The author can use
\verb|\classified{Not Classified}| to explicitly indicate that the
document is unclassified, but usually the macro should simply not be
called.

ITDS-style coversheets indicate whether ITDS should obtain approval prior
to releasing the document to AT\&T employees 
(excluding contract employees).  The default is permit release
without approval.  To require ITDS to obtain approval for each
release, use
\begin{itemize}
\item[]
\verb|\itdsrestrict|
\end{itemize}
\verb|\itdsrestrict| is automatically invoked when restricted or registered
proprietary levels are specified.
An organizational approval section is also included automatically 
when a restricted or registered proprietary level has been
specified.  In other cases such an approval section is optional and
may be requested by 
\begin{itemize}
\item[]
\verb|\organizationalapproval|
\end{itemize}
Or one may specify the names of approvers using from one to six calls to
\begin{itemize}
\item[]
\verb|\approver{|{\it name}\verb|}|
\end{itemize}
This also forces an approval section and guarantees enough signature
lines for those specified and prints the names under the lines,
similar to that done for authors.

\subsection{Location of the Data Macros}
All of the above macros for entering data, except \verb|\copyto| and
\verb|\coverto|  should be placed in the
preamble; i.e., before the \verb|\begin{document}| command.  The
\verb|\copyto| and \verb|\coverto| macros should be called after the
\verb|\makesignature| macro (see the discussion in the next section).
\section{Generating the Heading, Abstract, Cover Sheets, \dots}
The heading containing the AT\&T logo, title and author's
identification is placed at the top of the first page by calling
\begin{itemize}
\item[]
\verb|\makehead|
\end{itemize}
immediately after the \verb|\begin{document}| statement.  If the
abstract is to be printed on the first page, a call to
\verb|\makeabstract| should appear next.  The company name defaults to
AT\&T Bell Laboratories, but can be changed to another company name
but calling
\begin{itemize}
\item[]
\verb|\companyname{|{\it company-name}\verb|}|\\
\verb|\capcompanyname{|{\it company-name-in-all-caps}\verb|}|.
\end{itemize}
The later is only needed if an ITDS-style coversheet is generated where the
company name appears in all uppercase characters.

Following the body of the memorandum should be a call to one of the
signature macros
\begin{itemize}
\item[]
\verb|\makesignature|\\
\verb|\makefacesignature{|{\it face-root}\verb|}|
\end{itemize}
to place the standard signature information at the end.  The
\verb|\makefacesignature| version places a picture of the author's
face just to the left of his signature block.  To use this version
you must have a face.tex for each author in the file {\it
face-root}/{\it author-initials}/face.tex.  For example my face.tex is
in {\it face-root}\verb|/TLA/face.tex|.  Note that each author's
directory name must 
be the same as the string specified in \verb|\initials|.  The
number of faces that can appear on one page is limited by the size of
your \TeX\ data structures.  My system can only print two faces on
a page.  So for more than two authors, either the signatures must lie
across a page boundary or I must use the faceless version of
\verb|\makesignature|.  For more information on face and other bitmap
files see the section on bitmap.sty.

The signature should be followed by calls to the \verb|\copyto| and
\verb|\coverto| macros, if desired (see discussion of these in the
section on memorandum data).

If a list of references is desired the standard {\it thebibliography}
environment should be used.  By default the references are printed on
a separate page.  A new environment {\it thebibliography*} can be used
to print the reference list without starting a new page.  \BibTeX\ can
be used to create the bibliography.  There is an asterisk form,
\verb|\bibliography*{|{\it bib-files}\verb|}| that may be used to
supress the starting of a new page for the reference list when using
\BibTeX .  There is no default for
bibliography style.  The user must set one of the standard
bibliography styles or a new bibliography
style, {\it atttm.sty}.  The {\it
atttm} style is similar to the {\it unsrt} style but is somewhat closer to
the style recommended in {\it The Bell Labs Style
Guide}\cite{bib:style} (though there are still some differences).
The bibliography style can be set by calling
\begin{itemize}
\item[]
\verb|\bibliographystyle{|{\it style-file}\verb|}|
\end{itemize}
just after the \verb|\begin{document}|.

If appendices are used, a call to the new macro \verb|\appendices|
should occur after the bibliography (see discussion of the macro in
section \ref{sec:appendicies}).
  
Just as with the standard {\it article} 
style, table of contents,
list of figures and list of tables can be generated by calls to 
\begin{itemize}
\item[]
\verb|\tableofcontents|\\
\verb|\listoffigures|\\
\verb|\listoftables|
\end{itemize}

An ITDS-style coversheet in a form similar to that generated by the
docgen and docsend programs can be generated by the macro
\begin{itemize}
\item[]
\verb|\coversheet|
\end{itemize}
The coversheet macro should be called after {\it all} other items
(such as table of contents, \dots) to get the page counts correct.
It should appear just before the call to \verb|\end{document}|.  

The page counts are automatically made and printed on the coversheet.
If pages are to be inserted into the document that are not printed
with \LaTeX , one must correct the page numbers.  This can be done in
two ways.  If the pages appear after the last regularly numbered page;
i.e., after any references and appendices, then only the page count on
the cover sheet need be corrected.  This can be done with the macro
\begin{itemize}
\item[]
\verb|\extrapages{|{\it number-of-extra-pages}\verb|}|
\end{itemize}
If the extra pages occur in the middle of the sequence of numbered
pages; such as, in an appendix other than the last appendix, then the
page number used by \LaTeX\ must be corrected.  This can be done with 
\begin{itemize}
\item[]
\verb|\clearpage|\\
\verb|\addtocounter{page}{|{\it number-to-be-added}\verb|}|
\end{itemize}
This should be done seperately at each location where pages are to be
inserted. 

A memorandum-for-file-style coversheet (which has a heading like the
first page of the document but contains only the abstract) can also be
generated by the macro 
\begin{itemize}
\item[]
\verb|\mmfcoversheet|
\end{itemize}
The macro can be called before the first page of the document (but
after the \verb|\begin{document}| but will then be counted as page 1
with the ``first page'' begin numbered page 2.  It is best to call the
macro at the end, just before the call to \verb|\end{document}| as
with the ITDS-style coversheet and then move it to the beginning after
printing.
\subsection{Example Using {\it tm} to Generate a Complete Document}
An example source file for generating a complete document,
illustrating most features and including a coversheet is given in
Appendix A.  The resulting document is included in Appendix B.
For a more complete example, the source for {\it this} document is
available from the author or in the file \verb|tm.doc| in the tm
distribution. 
\section{Released Paper Style {\it rp.sty}}
A related style called rp (for released paper) is also available.
This style will accept all the macros used by {\it tm.sty} but produce a
document in the {\it released paper} style of MM.  Most of the tm
macros are simply ignored since they refer to things not included in
released papers.  They are still tolerated so that a single source
file can produce documents in both forms.  The only changes that must
be made are to change the \verb|\documentstyle{tm}| to
\verb|\documentstyle{rp}| and to define a ``city-state-zip'' address
using the macro
\begin{itemize}
\item[]
\verb|\cityaddr{|{\it city, ST  ZIP}\verb|}|
\end{itemize}
This information is used beneath the author's name and company
affiliation in the released paper style.  MM obtains this
automatically from a built-in table indexed by the location code.  A
future version of {\it rp.sty} may also contain such a table.  The
current version of {\it rp.sty} does {\it not} allow different
cityaddr for each author.  The \verb|\cityaddr| macro is tolerated by
the {\it tm.sty} but ignored, so if put in the document source file
the \verb|\documentstyle| alone determines the output form.

The proprietary markings are still printed in the released paper
style if included in the document source file.  MM does the same.
If the released paper is not proprietary the internal TM shouldn't be
either.  

The word ``ABSTRACT'' is added above the abstract printed on the
released paper form of the coversheet but not anywhere else the macro
\verb|\makeabstract| is called, such as on the first page.  This
allows the word or any other notation to appear above the abstract at the
discretion of the author.

\section{Extensions to \LaTeX}
\subsection{New Page Styles}
Some additional page styles have been added.  The style {\it
headandfoot} is similar to the standard \LaTeX\ style {\it myhead}
but adds a centered foot generated by the macro \verb|\bottommark|. 
Headandfoot is the default page style for the {\it tm} document style.
The bottommark can be manually set by calling
\begin{itemize}
\item[]
\verb|\markbottom{|{\it string}\verb|}|
\end{itemize}
The bottommark is set automatically by calls to the \verb|\markdraft|
and proprietary markings macros.  The standard macros for setting
headings are supported, \verb|\markright| and \verb|\markboth|.

A page style with the simpler heading consisting of only ``- pageno -''
(the default with the mm package) but the same foot is also available.
To use this style for the entire document specify
\begin{itemize}
\item[]
\verb|\pagestyle{simpleheadandfoot}|
\end{itemize}
in the preamble.

There is also an {\it onlyfoot} page style that has the same bottom
marking but no page headings.
\subsection{Enumerated Lists}
The mm macro package for troff allows ``enumerated'' lists using Roman
numerals and alphabetic letters in addition to Latin numbers.  This
is especially useful for outlines.  To support this, four new
environments have been added:  {\it romanenum, Romanenum, alphenum} and
{\it Alphenum}.  These are used exactly like {\it enumerate} but
use lower case Roman numerals (i, ii, iii, iv, \dots), uppercase
Roman numerals (I, II, III, IV, \dots), lower case letters and upper
case letters, respectively.
\subsection{Labeled Lists}
A new type of list, not available in mm, is available in tm.sty.  This
list environment is called {\it labeled} and has {\bf left} justified
labels and takes the indentation for the 
item body as a parameter.  For example the following commands
\begin{itemize}
\item[]
\verb|\begin{labeled}{1in}|\\
\verb|\item[label 1]This is the body of the item.  If the body continues|\\
\verb|onto a second line, the continuation will be indented like the|\\
\verb|body part of the first, to the amount specified in the parameter.|\\
\verb|\item[label 2]This is a second item, included to illustrate the|\\
\verb|resulting format.|\\
\verb|\end{labeled}|
\end{itemize}
would give the following results
\begin{labeled}{1in}
\item[label 1]This is the body of the item.  If the body continues
onto a second line, the continuation will be indented like the
body part of the first, to the amount specified in the parameter.
\item[label 2]This is a second item, included to illustrate the
resulting format.
\end{labeled}
\subsection{Inline Lists}
A new kind of ``list'' environment, {\it inlinelist},  has been added.
This formats a 
numbered list in-line in the paragraph; i.e., each item is labeled
with a number in parenthesis.  It is really not a
list environment in the \LaTeX\ sense, since it is not implemented as a
display and each item does not begin on a new line.  The desired
inter-item punctuation should be supplied at the end of each item.
For example, 
\begin{inlinelist}
\item one can use a semicolon as after this item;
\item a comma, or
\item place a period at the end.
\end{inlinelist}
This list was implemented with the following \LaTeX\ code
\begin{itemize}
\item[]
\verb|\begin{inlinelist}|\\
\verb|\item one can use a semicolon as after this item;|\\
\verb|\item a comma, or|\\
\verb|\item place a period at the end.|\\
\verb|\end{inlinelist}.|
\end{itemize}

\subsection{Draft Marking}
A document will be labeled as {\it DRAFT} at the bottom of each page
if the macro
\begin{itemize}
\item[]
\verb|\markdraft|\\
\verb|\markdraft*|
\end{itemize}
is called.  The `*' version adds the date and
time\markedfootnote{\dag}{Date and time versions contributed by Dennis
DeBruler}.  If a proprietary marking is also used \verb|\markdraft|
must be called {\it before} the \verb|\markprivate|,
\verb|\markproprietary|, \verb|\markrestricted|, or \verb|\markregistered|.
\subsection{Proprietary Markings}
Proprietary markings of the standard form are available by making
calls to 
\begin{itemize}
\item[]
\verb|\markprivate|\\
\verb|\markproprietary|\\
\verb|\markrestricted|\\
\verb|\markregistered|\\
\verb|\marknone|
\end{itemize}
Calls to any of these macros replace the bottommark and so replace
any bottommark set with \verb|\markbottom| or any previous call to a
proprietary marking macro.  \verb|\markprivate| also replaces the
heading on all pages with a centered, bold ``{\bf PRIVATE}'',
replacing the line with usually carries the title or markright
string.  \verb|\marknone| cancels a previous mark.
\subsection{Trade Marks}
The trademark symbol `\tm ' and registered trademark symbol `\regmark ' 
can be generated by the macros
\begin{itemize}
\item[]
\verb|\tm|\\
\verb|\regmark|
\end{itemize}
The registered trademark, \UNIX\ can be generated by 
\begin{itemize}
\item[]
\verb|\UNIX|
\end{itemize}
Note that a footnote indicating the ownership of the trademark is
automatically generated the first time the trademark is used in a
document.\markedfootnote{\dag}{See the example here.}

Other registered trademarks can be used with ownership footnotes by
the macros
\begin{itemize}
\item[]
\verb|\regnotice{|{\it ownership-text}\verb|}|\\
\verb|\tmnotice{|{\it ownership-text}\verb|}|
\end{itemize}
\subsection{Footnotes with Symbols}
An additional footnote style is supported, using symbols for the
footnote mark.  The macro is
\begin{itemize}
\item[]
\verb|\markedfootnote{|{\it symbol}\verb|}{|{\it footnote text}\verb|}|
\end{itemize}
\subsection{Time of Day}
The time of day can be printed by calling 
\begin{itemize}
\item[]
\verb|\hhmm|\\
\verb|\hhmm*|
\end{itemize}
The `*' version prints the time using the 24 hour
format.\markedfootnote{*}{24 hour version, contributed by Nelson Beebe in
texhas.86.002.  I modified it to offer a 12 hour version.}  Both of
these use the basic \verb|\time| macro of \TeX\ that gets the time of
day in minutes since midnight.
\subsection{Cent Sign}
The symbol for cents is available.  The macro suggested by Knuth on
page 319 of the TeXbook is the one used.
\begin{itemize}
\item[]
\verb|\cents|
\end{itemize}
\subsection{Appendices}\label{sec:appendicies}
The new macro 
\begin{itemize}
\item[]
\verb|\appendices|
\end{itemize}
should be used to begin the appendices.  It calls the
\verb|\appendix| macro which causes sections to be labeled with
capital letters (A, B, \dots) rather than numbers.  It also calls
\verb|\newpage| to begin the first appendix on a new page and
\verb|\addcontentsline| to make a table of contents entry labeled
``APPENDICES''.  
\subsection{Variable Width hlines in Tabular Environment}
The \LaTeX\ tabular environment has a parameter
\verb|\arrayruleswidth| that specifies the width of lines generated
with \verb|\hline, \cline|, and \verb|\vline|.  The value may be
changed outside the tabular environment (affecting subsequent
tabular environments) or it can be changed inside a single item of the
environment (one column entry of one line) (affecting the vertical
line, if any, that follows that column entry but {\bf not} subsequent
items).  There appears, however, to be no way to change the value to
effect a single horizontal line.  The implementation of \verb|\hline|
will not allow any macro calls between the \verb|\\| and the
\verb|\hline| so the value cannot be changed in the scope of the
horizontal line.

A new macro \verb|\varhline| has been added to allow this capability.
It can be used wherever \verb|\hline| can be used but takes a
parameter specifying the line width.
\begin{itemize}
\item[]
\verb|\varhline{|{\it line-width}\verb|}|
\end{itemize}

\section{Known Limitations of {\it tm} style}
\begin{enumerate}
\item Attachments are not automatically indicated on the signature
page as prescribed in the {\it AT\&T Bell Laboratories Office Guide}.
These may be added manually as illustrated in the example in
appendix A.
\end{enumerate}
\section{Other \TeX\ related Items}
\subsection{AT\&T Logo}
A few other items of interest to \TeX\ users have been developed. 
These are not a part of the {\it tm} style.  A font file containing
the AT\&T ball and the letters `A', `T', and `\&' in the AT\&T logo
style have been created using METAFONT.  Fonts have been generated
for Imagen (8/300) and Epson\tm\ LQ1500 (dot-matrix) printers.  
The METAFONT `.mf' file can be supplied to those who would like 
to generate the fonts for other printers.
\subsection{Extended Picture Environment}
Two extensions to the {\it picture} environment have been
developed, to provide features available in {\it pic}.  A circle
with text centered inside can be produced with the command
\begin{itemize}
\item[]
\verb|\labeledcircle{|{\it x-coord}\verb|}{|{\it y-coord}\verb|}{|{\it
  diam}\verb|}{|{\it string}\verb|}|
\end{itemize}
Note that the {\it x-coord} and {\it y-coord} are specified as
ordinary \TeX\ parameters {\it not} in parentheses separated by
commas.  

Pic allows circles to be easily connected by lines and arrows.  In
the {\it picture} environment lines and vectors are available, but
only at a few slopes and the end points must be known.  Thus to draw
an arrow from the perimeter of a circle, the
user must calculate the point that would lie on the perimeter in the
direction of the line.  A macro has been added that does this
calculation for the user, for all available slopes.  The command is
\begin{itemize}
\item[]
\verb|\arrowcircle{|{\it x-coord}\verb|}{|{\it y-coord}\verb|}{|{\it
  h-slope}\verb|}{|{\it v-slope}\verb|}{|{\it circ-diam}\verb|}{|{\it
  dimen}\verb|}|
\end{itemize}
As with \verb|\labeledcircle| all parameters are given in the \TeX\
style.  The {\it x-coord} and {\it y-coord} are the coordinates of
the center of the circle from which the arrow is to be drawn.  The
{\it h-slope}, {\it v-slope} and {\it dimen} are given in the same 
way as for \verb|\vector| except that {\it dimen} is the distance
from the center of one circle to the {\it center} of another circle.  
The arrow will actually be
truncated at the perimeter of the other circle.  The {\it circ-diam} 
is the diameter of the circles.  The current version requires that
both circles be the same diameter.  Note that neither of the circles
are drawn by the command.  The user must still locate the circles in
such as way as to use one of the available slopes.  A way of
specifying a series of circles and arrows that are simply
concatenated but in which the user does not specify the center
coordinates is being investigated.

To use these two macros the file {\it expicture.sty} must be
available and it must be loaded by adding `[expicture]' as an option
on the \verb|\documentstyle| command.
\subsection{Difference Marking}
A difference marking filter, {\it ldiffmk}, similar to the standard \UNIX\
utility {\it diffmk}, is available for use with \LaTeX .  
{\it Ldiffmk} is a simple modification to {\it diffmk} and has the same limitations
that {\it diffmk} has as well as a new one. 
\begin{enumerate}
\item Since {\it diffmk} and {\it ldiffmk} are line-oriented,
detecting lines that are not identical, a simple change the location
of a source-file line break will be detected as a change even though
the output of \TeX\ will be identical.
\item Differences are indicated by a `T' in the right margin where
the change begins and by an upside-down `T' in the right margin
where it ends.  It does {\it not} connect them with a solid line as
{\it diffmk} with {\it troff} does.  It does this by using the
marginal note feature of \LaTeX .  Single line changes must still be
indicated by marks on two lines and will cause a warning from \LaTeX\
that one mark has been moved.  
\item Marginal notes are not allowed in text that is used as an
argument for a macro, so differences in text for abstracts and
footnotes can cause \LaTeX\ to fail and the difference marks must be
manually removed.
\end{enumerate}
See the man page for {\it ldiffmk} for a more detailed discussion.  A
more robust version will be developed in the future.  In spite of
these limitations, {\it ldiffmk} can be a useful tool for reviewing
multiple revisions of a document.
\subsection{Bitmaps}
There is a {\it bitmap.sty} that can be included as an option (call 
\verb|\documentstyle[bitmap]{tm}|).  See section \ref{sec:bitmaps}
\subsection{Author Database}
To reduce the nuisance of reentering author related data into each
document, one may wish to collect the data for several authors into an
author database.  An example is included in the distribution called
authors.sty.  You should edit it to add the data for yourself and the
other authors you commonly coauthor papers with.  You can include the
database as an option in the \verb|\documentstyle| macro and then
enter the author related data into the document as
\begin{itemize}
\item[]
\verb|\authordata{|{\it author-code}\verb|}|
\end{itemize}
\section{Local Modifications to tm.sty}
Some sites who install tm.sty and its related style files may wish to
alter some of the features or defaults.  There are two files that may
be modified to allow local changes without modifying tm.sty itself.
The file {\it local.sty} is intended to contain page offset commands
to adapt for positioning differences in some printers.  It can also
contain commands to change the page sizes.  This file is included
before most of the tm definitions that differ from article.sty.  This
means that any references to dimensions in the tm macros {\it will} be
affected by changes made in local.sty.  However, macro definitions
placed here would be overridden by those in tm.sty.  If you wish to
alter any of the defined macros, do so in the file {\it localpatch.sty}.
This is included at the very end of tm.sty.  All changes where
possible should be made in these files rather than by editing tm.sty.
This will prevent the necessity of remodifying future releases of tm.sty.

\part{MM to \LaTeX\ translator,{\it mm2tex}}

The {\bf mm2tex} filter converts documents written in mm into \LaTeX\ using
the tm.sty style developed for writing AT\&T memoranda.  The filter
currently uses a combination of sed and awk scripts.  While it does
not handle all mm and troff macros, it handles the majority of those
used in typical memos.  Full documentation is given in the document
{\it An MM to \LaTeX\ Translator}.\cite{bib:mm2tex}

\part{Letter Style {\it attletter.sty}}
The {\it attletter.sty} is modeled after the standard \LaTeX\
letter.sty with a few modifications.  The AT\&T letter head will be
automatically printed at the top of the first page.  The
\verb|\companyname| macro can be used to change the company name from
the default, AT\&T Bell Laboratories.  The standard letter style
address macro can be used to define the return address.  

There is a predefined address macro that allows the following macros
\begin{itemize}
\item[]
\verb|\room{|{\it room-number}\verb|}|\\
\verb|\streetaddr{|{\it street-address}\verb|}|\\
\verb|\cityaddr{|{\it city-state-zip}\verb|}|\\
\verb|\phone{|{\it full-phone-number}\verb|}|\\
\verb|\fax{|{\it fax-number}\verb|}|\\
\verb|\eaddress{|{\it electronic-address}\verb|}|
\end{itemize}
to specify the room number, street and city address, a personal phone
number, fax number and an 
electronic address that can optionally be called.  All items have
defaults (some default to blank) appropriate for the Liberty Corner
location, so only those values you want to override need be specified.
Defaults can be replaced with blanks by calling the macro with a null
argument.  It is recommended that the style file 
be edited by the one who installs it to default to values appropriate
for the site where the file is installed, so that 
each user does not need to specify items like the the building mailing
address.

There is a new dimension parameter \verb|\labelheight| that specifies 
the height of the address labels printed by the standard declaration
\verb|\makelabels|.  The default is 1 inch rather than 2 inches as in
the standard letter.sty.  Labels are printed ``two up''; i.e., two across
the page, and spaced vertically by labelheight.  The height can be
changed to fit specific ``peel-off'' labels by 
\begin{itemize}
\item[]
\verb|\labelheight=|{\it height}
\end{itemize}

\verb|\makelabels*| causes a return address label is printed for each
address label.  The return address is predefined similarly to address
but without phone numbers and additional data.  The name given in the
signature macro is added to the top.  If the argument to the signature
macro includes data other than the name, the user can optionally call 
\begin{itemize}
\item[]
\verb|\name|{\it author-name}
\end{itemize}
to supply the name to be used.

Many of the extensions to \LaTeX\ discussed below are also available
in the attletter style.  There is a face version of the
\verb|\signature| macro
\begin{itemize}
\item[]
\verb|\facesignature{|{\it signature}\verb|}{|{\it face-file}\verb|}|
\end{itemize}
The signature field is the same as for the normal \verb|\signature|
macro.  The face-file is the face file name and any necessary path not
simply a face root directory as for the tm.sty.

\part{A Voucher Style, {\it voucher.sty}}
\section{Creating Vouchers}

There is a vouchform.tex that can be used as a template for a voucher.
The voucher must be processed by LaTeX using voucher.sty.  There is a
script `voucher' that uses a makefile (called vouch.Makefile in the
original distribution).  Voucher can be used to automate the various
processing steps after a copy of vouchform.tex is filled out.

Note that the parameters of \verb|\expense| are positional parameters in the
same order as the columns in the expense voucher.  A comment line at
the top documents this order.  A blank expense line is included as a
comment.  Make copies of this line for each expense line desired,
remove the comment `%' and fill out, adding amounts (in cents or like
normal but omit the decimal!) and leaving any columns empty for which
there was no expense.  Horizontal and vertical totals will be computed
for you.  Notice that the automobile expenses you may put in the
dollar amount or use the optional \verb|\miles| macro (see second example
expense line) to put in number of miles and allow macro to calculate
dollar amount.

Data for the top part of the form is in the preamble part of the .tex
file and has been seperated (in vouchform.tex) into data that will be
constant for most users and data that will change with each voucher.
You can fill out the fixed part and then save the file as a personal
vouchform template.  Note that \verb|\amexexp| is for total amount charged to
Corp Am Express card (card number given in \verb|\amexnum|).
\verb|\amexadvance|  is
for total of Am Express advances.  \verb|\expensetype| checks one of the
boxes at the top of the form.  Currently there is no way to Foreign
Travel \$ rate blank.  Note that on current forms there is no box
numbered 3.

The otherexpenses environment is for the Other Expenses part of the
form.  Again there is a commented template line and one documenting
the parameter order.  The items must also be included in the other
column of the \verb|\expense| lines -- this cannot be automatic since the
form cannot determine which expense line the item should go on and
often several \verb|\otherexpenses| are totaled on one \verb|\expense| line.  

the projnums environment is for the project number part of the form.
the first paramenter is for the project number and the second for the
percentage to be charged to that project number.  The sty verifies
that the percentages add up to 100 or prints and error to stdout.  I
find it convenient to put lines for each project number I use into my
personal vouchform.tex and comment out the lines not used for a given
voucher.  

\section{Voucher Script}

{\it voucher*} are files for preparing vouchers.  

Because vouchers are printed landscape style, you probably need
postscript to print them.  I use dvips.  

I find it convenient to set up makefile for documents.  vouch.Makefile
is a makefile for vouchers.  I place all vouchers in a voucher
directory with vouch.Makefile.  The makefile uses DOC as a variable.
I use the script voucher to pass this in.  I use a similar make file
for other documents by placing such documents in their own directory
and setting DOC in the Makefile.

Note that the method for printing postscript files often varies with
different sites.  One common command uses prt.  The script voucher is
setup to use prt.  If this is NOT appropriate at your site, define the
environment variable PSPRINT as the appropriate command taking the
postscript file as the parameter and export it.  For example:
\begin{itemize}
  \item[]
        \verb|$ export PSPRINT=/usr/local/bin/psprint|
\end{itemize}

If you have trouble check voucher and vouch.Makefile for other site
dependent locations.  VOUCHBIN can be defined as the path to latex and
dvips if these are not in the PATH.
\begin{itemize}
  \item[]
        \verb|$ voucher |{\it voucher-name} \verb|[dvi|ps|print|printdoc|test|exp-test]|
\end{itemize}

voucher calls a makefile that does a 
\begin{itemize}
  \item[]
        \verb|$ latex voucher-file|
\end{itemize}
if voucher-file.dvi is not up to date, then a converts to PostScript
if voucher-file.ps is not up to date.

\noindent OPTIONS
\begin{description}
  \item[dvi]will force the make to stop with a dvi file NOT
	making a PostScript file.
  \item[ps]the default
  \item[print]make sure that voucher-name.ps is up to date and then print
	the file.
  \item 
\end{description}

(There are some additional options in vouch.Makefile.)

Note that vouch.Makefile uses a PSPRINT string that contains the
command to print on a postscript printer.  I use a script called
psprint.  Some can simply use lp.  Edit this as appropriate.


\section{Using Vouchers Prepared with voucher.sty [updated June 1991]}

I wrote to Treasury including a sample of my voucher and asked if they
would accept them.  I never received a reply.  After a month I just
started submitting them and have never gotten a complaint.  They have
been accepting them from 3 or 4 of us for about 5 months, at least
10-15 vouchers total, and none of use have had any problems.  One user
apparently got a question -- they were puzzled as to why they weren't
blue, but it was accepted.

I should point out that we do not have a BL cashier in our building so
all our forms have been sent directly to Treasury.  I do not know if
one might have more trouble at a cashier counter.

\part{SLi\TeX\ Style for AT\&T Viewgraphs}
SLi\TeX\ is a standard \LaTeX\ system for making viewgraphs.  It is
documented in a chapter in the \LaTeX\ book\cite{bib:latex}.  SLi\TeX\
input consists of two files, called the {\it root file} and the {\it
slide file}.  The root file contains a standard \verb|\documentstyle|
line and a document environment.  Inside the document environment is
either a \verb|\colorslides{|{\it slidefilename}\verb|}| or a
\verb|\blackandwhite{|{\it slidefilename}\verb|}| call to indicate the
slide file.  The slide file contains slide, note and overlay
environments producing pages of the corresponding type.

The standard SLi\TeX\ documentstyle is called {\it slides}.  We offer
an alternative slide style, {\it attslides}, that produces slides in
one of the commonly used AT\&T specific styles.  It uses 17 pt sans
serif font for the body of each slide including itemize lists and uses
29 pt italic san serif font for headings.  Headings appear above a
horizon line that has an 11 pica underline or multiweight line similar
to that used in Multiweight Design.  The corporate signature (AT\&T
logo) appears at the bottom right.

Slides (in a dvi file) are produced with the slitex command (rather
than the latex command). 
\begin{itemize}
\item[]
\verb|$ slitex |{\it rootfilename}
\end{itemize}

Slides are sized for 11 in by 8.5 in paper in landscape mode rather
than the usual portrait mode of slides.sty.  This means that they
{\it cannot} be printed with most dvixxx printer filters.  The pages
can be rotated, however, by most PostScript\regnotice{PostScript is a
trademark of Adobe Systems Incorporated} printer filters, such as
dvips.  The usual invocation is
\begin{itemize}
\item[]
\verb|$ dvips -t landscape |{\it rootfilename}
\end{itemize}
The resulting file can then be printed in normal way for PostScript files.

\part{Document Style for Multiweight Design, {\it mw.sty}}
Multiweight Design, or officially {\bf
AT\&T Documentation Architecture}\cite{bib:mwstd}, is a set of standards
for the content and format of documentation intended for ``all
postsale, external-customer documents typically delivered with or in
support of a product or service''.  The standards are optional for
presale information, internal documents (such as requirements and
design specifications) and training materials.

The standards are intended for large documents divided into chapters
and various sections with headings and a separate title page.  

\section{The Multiweight Design Style} The multiweight design style is
a \LaTeX\ style used with \TeX\ and \LaTeX\ to produce documents.  It
uses most of the same macros as {\bf tm.sty} and so it
is possible to easily change from one style to the other to change the
printed format of a document.  There are some limitations.
Multiweight Design is a standard format for multichapter books and
mw.sty is based on the standard book.sty of \LaTeX .  Tm.sty is
intended for technical memoranda and is based on article.sty.  There
is no chapter macro in article.sty or tm.sty and so a tm.sty document
could be incorporated into a mw.sty book as a chapter with the
addition of a chapter macro.  Taking a mw.sty book and reformatting it
as a TM would take more revision.

The style is fully documented in the document {\it A \LaTeX\ Style for
Multiweight Design} which also serves as an example.

\part{Style Options}
Some additional style options are available.  Unless otherwise noted
these work with all of the styles discussed in this document and with
most \LaTeX standard styles.
\section{Conditional Text}
A style option, {\t condition.sty}, to support {\it conditional text} is
available.  {\it Conditional text} is text that should sometimes be
included in the document and other time not included or included in
certain versions of a document and not in others.  Condition.sty
supports multiple condition tags that can be independently turned off
or on.  These behave something like \#ifdef's in C code. 
The document style option {\it condition} must be added to the documentstyle
line.   New conditional tags are
declared using
\begin{itemize}
\item[]
\verb|\newconditional{|{\it tag}\verb|}|
\end{itemize}
Text is marked as conditional by using
\begin{itemize}
\item[]
\verb|\begin{conditional|{\it tag}\verb|}|\\
\verb|   Some Text|\\
\verb|   ...{}|\\
\verb|\end{conditional|{\it tab}\verb|}|
\end{itemize}
where tag is one of the declared conditional tags.  By default
conditional text is included in the document.  To hide the 
conditional text, call
\begin{itemize}
\item[]
\verb|\conditional|{\it tag}\verb|off|
\end{itemize}

\section{Bitmaps\label{sec:bitmaps}}
There is a {\it bitmap.sty} that can be included as an option (call 
\verb|\documentstyle[bitmap]{tm}|).  This allows small bitmaps to be
printed by the following
\begin{itemize}
\item[]
\verb|\input| {\it bitmap-filename}\\
\verb|\vbox{}\hskip2in\input| {\it bitmap-filename}
\end{itemize}
As with all input files the bitmap-file name must end in .tex, but the
.tex is not used in the command.  The later form illustrates how to
print the bitmap 2 inches from the left margin.  The size of \TeX\
data structures severely limits the number and size of bitmaps that
can be printed on a single page.  My implementation limits me to about
100 by 100 or two face files of 48 by 48.  Included with the system is
a utility, {\it bitmap2tex}, written in awk for converting X-window
bitmaps into a form compatible with this macro.  There is also a c++
program, {\it makeface}, for converting 9$^{th}$ Edition \UNIX\ face
files to X-window bitmaps.  By using {\it makeface} and {\it
bitmap2tex} one can obtain faces from the 9$^{th}$ Edition face server
and convert them into a form for printing with this macro.  There is a
special version of the makesignature macro, \verb|\makefacesignature|
in tm.sty that places the author's faces just to the left of their
signature blocks at the end of a document.  For this macro all of the
author's faces must be stored in the following directory structure
{\it some-face-home/authors-initials}/face.tex.  There is no
documentation at this time on how to make your own bitmap-file.tex's,
but two examples, square and tlaface, are included in both X-window
bitmap and tex forms.  If you have this memo in \TeX\ source form, you
can try the macro by creating a directory TLA in the directory where
the source is stored moving the file tlaface.tex to the new directory
and renaming it face.tex, adding \verb|[bitmap]| to the
\verb|\documentstyle| and changing \verb|\makesignature| to
\verb|makefacesignature{|{\it directory-of-this-document}\verb|}| and
rerunning it through \LaTeX .

\part{Availability}
All styles were developed for \LaTeX\ version 2.09 and should
be compatible with that and later versions on any machine that has
\TeX\ and \LaTeX .  The author uses them on Sun Workstations\regnotice{Sun
Workstation is a registered trademark of Sun Microsystems, Inc.}
(3/110 and 4/110) and
on a VAX\regnotice{VAX is a trademark of Digital 
Equipment Corporation} (running \UNIX\ System V.2) and prints on an
Imagen 8/300 laser printer.  He also has
used some of them on an AT\&T 6300plus printing on an Epson\tm\
LQ-1500 dot-matrix 
printer.  No changes should be required to use the package on a
different computer.  The only change that should be required to 
use it with a different printer is to build the font files for the
AT\&T logo.  METAFONT would be required to build these font files.  
The author can supply the font files for Imagen 8/300 so for that
printer METAFONT would {\it not} be required. 

All files are available on request from the author.  Send email to
tla@bartok.att.com.
\makesignature

\noindent Atts.\\
References\\
Appendixes A - D

\medskip
\copyto{{\nobreak\topsep=0pt\partopsep=0pt\begin{tabbing}\nobreak%
\hskip2in\=WW \=\kill%
Y.\ Afek		\>MH 3D-438\\
E.\ Ayanoglu (vax135!ender)\>HO 4F-507\\
M.\ Beutnagel		\>MH 2D-513\\
C.\ D.\ Blewett		\>MH 2C-470\\
D.\ S.\ Booth		\>HO 4E-526\\
R.\ J.\ Brackman	\>MH 3C-439\\
B.\ Carpenter (ho5cad!wjc)	\>HO 1L-410\\
M.\ Carroll		\>LC 4N-D02\\
H.\ H.\ Chefitz		\>LC 3W-F09\\
Y.\ Chen		\>MH 3C-535\\
D.\ L.\ DeBruler	\>IHP 1F-114\\
R.\ A.\ Deighan		\>ALC 1C-343\\
P.\ Glassman		\>HO 3K-324\\
R.\ S.\ Hauck		\>IH 2E-424\\
A.\ I.\ Hauser		\>MT 3D-205\\
K.\ H.\ Horton		\>PR 5-3047\\
Dan Jacobson		\>IH 1D-213\\
T.\ J.\ Kowalski	\>MH 2C-552\\
P.\ Kravitz		\>MH 30H-016\\
Bala Krishnamurthy	\>LC 4N-E03\\
G.\ R.\ Kuntz		\>SF 5345\\
L.\ S.\ Levy		\>MT 3G-108\\
M.\ Y.\ Liberman	\>MH 2D-444\\
J.\ P.\ Linderman	\>MH 3D-435\\
Bill Mershon		\>LC 4N-E08\\
Victor McCrary		\>MH 2A-337\\
M.\ Murdocca		\>HO 4G-538\\
R.\ B.\ Murray		\>LC 4N-W09\\
Steve North (ulysses!north)	\>MH 3C-539\\
Kostas Oikonomou (houdi!ko)	\>HO 3M-521\\
Jong Park (speedy!jtp)	\>SK 527\\
A.\ Reibman		\>HO 2L-518\\
Larry L.\ Rose		\>LC 4N-E01
\end{tabbing}}}%\vspace*{-20pt}
\restofcopyto{{\nobreak\topsep=0pt\partopsep=0pt\begin{tabbing}\nobreak%
\hskip2in\=\kill
J.\ R.\ Rowland		\>LC 3W-G25\\
E. Sampieri		\>HO 4G-616\\
Ravi Sharma		\>HR 2K-115\\
T.\ Sizer		\>HO 4G-530\\
Richard Stone (hocus!res)	\>HO 3K-328\\
V.\ Tavernini		\>AN 4S-125\\
S.\ K.\ Tewksbury (vax135!skt)	\>HO 4B-505\\
H.\ Trickey (coma!trickey)	\>MH 2C-460\\
J.\ D.\ Weld		\>WH 4C-228A\\
Wayne Wolf (allegra!wolf)	\>MH 3D-474\\
Anthony Zawadzki	\>HO 4K-406 
\end{tabbing}}}
\coverto{
All Supervision Lab 5911\\
All Members of Department 59112\\
All Members of Department 59114}
\begin{thebibliography}{99}

\bibitem{bib:latex}Leslie Lamport.  1986.  {\it \LaTeX : A Document
Preparation System}.  Reading, MA: Addison-Wesley.
\bibitem{bib:mm}D.\ W.\ Smith, J.\ R.\ Mashey, E.\ C.\ Pariser \& N.\
W.\ Smith. ``MM -- Memorandum Macros'', \UNIX\ document 1098.
\bibitem{bib:mm2}{\it \UNIX\ System Document Processing Guide}.
\bibitem{bib:style}April Cormaci and Richard Trenner.  1988. {\it The Bell
Labs Style Guide}, AT\&T Bell Laboratories, Kelly Education and
Training Center. February.
\bibitem{bib:mwstd}AT\&T. 1990. {\it AT\&T Documentation Architecture,
Content and Format Standards}. Document number AT\&T 000-110-000
(available from Customer Information Center 800 432-6600).
\bibitem{bib:mw}Terry L Anderson.  1991. {\it A \LaTeX\ Style for Multiweight
Design}.  Technical Memorandum 59112-91xxxx-01TMS. AT\&T Bell Laboratories.
\bibitem{bib:mm2tex}Terry L Anderson.  1991. {\it An MM to \LaTeX\
Translator}. Technical Memorandum 59112-91xxxx-01TMS. AT\&T Bell Laboratories.
\end{thebibliography}
\appendices
\section{An Example Source File}
The following is an example source for a document showing many of
the available features.
\def\bv{\begin{verbatim}}% These are necessary to get verbatim to 
\def\infil{\input tmexample}% begin only after input is redirected
\expandafter\bv\infil
\end{verbatim}
\clearpage
\section{The Resulting Example Document}
The following pages show the document that is produced by the
source file in the previous appendix. 
\clearpage\addtocounter{page}{8}% 8 pages of example added
\section{An Example Letter Source File}
The following is an example source for a letter using attletter.sty
\def\bv{\begin{verbatim}}% These are necessary to get verbatim to 
\def\infil{\input letexample}% begin only after input is redirected
\expandafter\bv\infil
\end{verbatim}
\clearpage
\section{The Resulting Letter}
The following pages show the letter that is produced by the
source file in the previous appendix. 
\clearpage
\addtocounter{page}{4}% \clearpage added one and 3 more example pgs
\tableofcontents
\coversheet
\end{document}
%%%%%%%%%%%%%%%%%%%%%%%%%%%%%% THE END %%%%%%%%%%%%%%%%%%%%%%%%
