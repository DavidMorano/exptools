%
% EXAMPLE
%
\documentstyle{tm}
\markdraft*
\markrestricted
\classified{Secret}
\itdsrestrict
\title{A Sample Document that is Incredibly Long in Fact so Long that
it Will not Fit on a Single Line Even on the Coversheet but requires
at Least Two Lines}
\markright{A Short Version of the Title}
\typistinitials{typ}
\author{Terry L.\ Anderson}
\initials{TLA}
\eaddress{tla@bartok.att.com}
\department{00000}
\approver{M.\ Y.\ Head, Head 00000}
\location{LC}{9N-Z00}{9999}
\documentno{000625}{03}{TM}
\previousdocument{00000-000101-01-IM, 00001-000111-03-IM}
\author{Jane M.\ Smith-Fredrickson}
\initials{JMS}
\signatureextra{Department Head}
\department{00001}
\approver{U.\ R.\ Boss, Head 00001}
\location{AB}{2H-123}{1234}
\documentno{000625}{03}{TM}
\author{Longfirstname M.\ Lastname}
\initials{LML}
\eaddress{somemachine!fpj}
\department{00002}
\approver{Big Guy, Head 00002}
\location{AB}{2H-112}{2345}
\documentno{000625}{03}{TM}
\author{Mary~J.~Johnson}
\initials{MJJ}
\signatureextra{My Favorite Group\\A Really Great Department}
\eaddress{mjj@research.att.com}
\department{00001}
\location{AB}{2H-124}{3456}
\documentno{000625}{03}{TM}
\author{Not B.\ L.\ Employee}
\initials{NBLE}
\company{Some Other Co.}
\maddress{1234 Some St.\\Somewhere, XX\quad 00000}
\eaddress{...!ihnp4!hiscpu!nble}

\date{\today}
\chargecase{000000-0100}
\filecase{00001}
\filecase{00002}
\keywords{TeX; LaTeX; Memorandum; Document Preparation; Example}
\mercurycode{CMP}
\mercurycode{STD}
\memotype{TECHNICAL MEMORANDUM}
\abstract{
This small sample document illustrates many of features of the {\it
tm} document style.}
\begin{document}
\makehead
\makeabstract
\section{The First Section}
This is the first section.  To show more features let us refer to
the \UNIX\ operating system.
\subsection{A Subsection}
This is a subsection.  It will be numbered 1.1 and be included in
the table of contents.\markedfootnote{\dag}{This is a footnote
marked with a dagger.}

Note that the two hypenated names are printed differently.  Tm.sty
will not break a hyphenated name at the explicit hyphen.  So if you
use a tie (`\verb|~|') rather than a space between the parts of the
name it will force the entire name to appear on one line (even if it
exceeds the margin).  If you would like the name to be broken onto
more than one line, use normal spaces.  Remember that initials
followed by a `.' as with any abbreviation should use `\verb|\ |'
(backslash space) rather than simply space after the `.'.  Otherwise
\TeX\ considers it the end of a sentence and inserts a wider space
than indended.

We can include the time of day in a document as ``it is now \hhmm''.
Notice that this document has called \verb|\markdraft*| in the
preamble so the draft marks at the bottom of the page show the time of
day for this draft.  The \verb|\markdraft*| uses \verb|\hhmm*| to give
the time of day in 24 hour format.

Here we include an enumerated list using capital alphabetic letters.
\begin{Alphenum}
\item The first item
\item The second item.  Let us make it long enough to 
take more than a single line.  Do you think that this is enough?
\item The last item.
\end{Alphenum}
\section{The Second Section}
This is another section.  We will refer to the \LaTeX\ manual to
illustrate references.\cite{bib:latex}  Note that the actual
reference data is included in the {\it thebibliography} environment
after the \verb|\coverto| list.  The {\it bib:latex} is the user
chosen label for referring to the bibliographic item.

\section{Last Section}
Note that the \verb|\signatureextra| after the \verb|author{|{\it
name}\verb|}| at the beginning of this document has two lines of
extra data; in this case identifying the author's group and
department.  This will be printed under the signature right after
this section.
\makesignature

\noindent Atts.\\
References\\
Appendixes A and B

\medskip
\copyto{
All Supervision Center 9999\\
M.\ T.\ Staff\\
D.\ M.\ T.\ Staff\\
M.\ Supervisor}
\coverto{
All Members of Department 99999\\
J.\ Smith\\
J.\ Doe}
\begin{thebibliography}{99}

\bibitem{bib:latex}Leslie Lamport, {\it \LaTeX : A Document
Preparation System}, Reading, MA: Addison-Wesley, 1986.
%\bibitem{bib:abbott}Russell J.\ Abbott, {\it An Integrated Approach
%to Software Development}, New York: John Wiley \& Sons, 1986.
\end{thebibliography}
\appendices
\section{An Appendix Title}
The body of the appendix.
\section{The Other One}
We will include another section to illustrate that appendices can still
have sections.  They can of course also have subsections, \dots

The sections can be used as separate appendices in which case it
might be appropriate to use \verb|\newpage| just before the
\verb|\section| to make each begin a new page.
\tableofcontents
\coversheet
\end{document}
%
% END of example
%
