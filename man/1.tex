


     TTTTEEEEXXXX((((1111))))            1111999999990000 SSSSeeeepppptttteeeemmmmbbbbeeeerrrr 22225555 ((((EEEEXXXXPPPPTTTTOOOOOOOOLLLLSSSS))))             TTTTEEEEXXXX((((1111))))



     NNNNAAAAMMMMEEEE
          tex - text formatting and typesetting

     SSSSYYYYNNNNOOOOPPPPSSSSIIIISSSS
          tttteeeexxxx [ first line ]

     EEEEXXXXPPPPTTTTOOOOOOOOLLLLSSSS VVVVEEEERRRRSSSSIIIIOOOONNNN
          This is a slightly modified (by Dan_Jacobson@ATT.COM)
          version of a TeX manual page from labrea.Stanford.EDU of the
          same date as above.  Rather than make extensive changes to
          it which will need to be integrated into new versions of
          this manual page from there or elsewhere, we exptools' TeX
          package providers maintain hints, tips, status commentaries,
          local bugs, and information on our extensions and
          differences from the standard TeX and LaTeX, etc.
          distributions in the directory $TOOLS/lib/tex/tips
          (TOOLS=`logdir exptools`).

          Please look there for various caveats, including items
          superseding this manual page.  Running TeX/LaTeX/SLITeX
          brings you most of the way to obtaining printed output, but
          the current method to make the last necessary steps is
          discussed in the "tips" directory.

          Note there are AT&T LaTeX "documentstyles" available, and
          the exptools' docsubmit(1) command accepts the TeX family of
          languages.

          If you have any questions about TeX/LaTeX/SLITeX etc. after
          looking in our "tips" directory, or perhaps can't find our
          tips directory, you can contact us.  To find out who we, the
          exptools tex package (TeX/LaTeX/SLITeX, etc.) providers,
          currently are, use the exptools' command "providers tex".
          As of this writing we were Mark_A_Hartman@ATT.COM and
          Dan_Jacobson@ATT.COM.

          LaTeX and SLITeX are versions of TeX with additional macros
          to make writing papers and slides easy, see the LaTeX book,
          reference below.

          One needn't worry about initex and virtex mentioned below.
          See the tips directory for more information on them if
          you're interested.

          We now continue with the rest of the Labrea man page.
          (Note: note location and settings of a few of the things
          below may superseded by what is mentioned in the
          aforementioned "tips" directory.  Again, we have only made
          minimal changes to the Labrea man page.)


     DDDDEEEESSSSCCCCRRRRIIIIPPPPTTTTIIIIOOOONNNN



     PPPPaaaaggggeeee 1111                                           ((((pppprrrriiiinnnntttteeeedddd 3333////9999////99991111))))






     TTTTEEEEXXXX((((1111))))            1111999999990000 SSSSeeeepppptttteeeemmmmbbbbeeeerrrr 22225555 ((((EEEEXXXXPPPPTTTTOOOOOOOOLLLLSSSS))))             TTTTEEEEXXXX((((1111))))



          TeX formats the interspersed text and commands contained in
          the named files and outputs a typesetter independent file
          (called _D_V_I which is short for _De_Vice _Independent).  TeX
          capabilities and language are described in _T_h_e _T_e_X_b_o_o_k by
          Donald E. Knuth, published by Addison-Wesley.

          TeX is normally used with a large body of precompiled
          macros, and there are several specific formatting systems,
          such as LaTeX, which require the support of several macro
          files.  The basic programs as compiled are called _i_n_i_t_e_x and
          _v_i_r_t_e_x, and are distinguished by the fact that _i_n_i_t_e_x can be
          used to precompile macros into a ._f_m_t file, which is used by
          _v_i_r_t_e_x.  On the other hand, _v_i_r_t_e_x starts more quickly and
          can read a precompiled ._f_m_t file, but it cannot create one.
          It is the version of TeX which is usually invoked in
          production, as opposed to installation.

          Any arguments given on the command line to the TeX programs
          are passed to them as the first input line.  (But it is
          often easier to type extended arguments as the first input
          line, since shells tend to gobble up or misinterpret TeX's
          favorite symbols, like backslashes, unless you quote them.)
          As described in _T_h_e _T_e_X_b_o_o_k, that first line should begin
          with a file name or a \controlsequence.  The normal usage is
          to say ``tttteeeexxxx _p_a_p_e_r'' to start processing _p_a_p_e_r._t_e_x. The name
          ``paper'' will be the ``jobname'', and is used in forming
          output file names.  If TeX doesn't get a file name in the
          first line, the jobname is ``texput''.  The default
          extension, ._t_e_x, can be overridden by specifying an
          extension explicitly.

          If there is no paper.tex in the current directory, TeX will
          look look through a search path of directories to try to
          find it.  If ``paper'' is the ``jobname'' a log of error
          messages, with rather more detail than normally appears on
          the screen, will appear in _p_a_p_e_r._l_o_g, and the output file
          will be in _p_a_p_e_r._d_v_i. The system library
          $$$$TTTTOOOOOOOOLLLLSSSS////lllliiiibbbb////tttteeeexxxx////mmmmaaaaccccrrrroooossss contains the basic macro package
          _p_l_a_i_n._t_e_x, described in _T_h_e _T_e_X_b_o_o_k, as well as several
          others.  Except when ._f_m_t files are being prepared it is
          hardly ever necessary to \input plain, since almost all
          instances of TeX begin by loading _p_l_a_i_n._f_m_t.  This means
          that all of the control sequences discussed in _T_h_e _T_e_X_b_o_o_k
          are known when you invoke _t_e_x.  For a discussion of ._f_m_t
          files, see below.

          Several environment variables can be used to set up
          directory paths to search when TeX opens a file for input.
          For example, the _c_s_h command
              setenv TEXINPUTS .:/usr/me/mylib:$TOOLS/lib/tex/macros
          or the _s_h command sequence
              TEXINPUTS=.:/usr/me/mylib:$TOOLS/lib/tex/macros



     Page 2                                           (printed 3/9/91)






     TTTTEEEEXXXX((((1111))))            1111999999990000 SSSSeeeepppptttteeeemmmmbbbbeeeerrrr 22225555 ((((EEEEXXXXPPPPTTTTOOOOOOOOLLLLSSSS))))             TTTTEEEEXXXX((((1111))))



              export TEXINPUTS
          would cause all invocations of TeX and its derivatives to
          look for \input files first in the current directory, then
          in a hypothetical user's ``mylib'', and finally in the
          system library.  Normally, you would place the variable
          assignment which sets up the TEXINPUTS environment variable
          in your ._l_o_g_i_n or ._p_r_o_f_i_l_e file.  The environment variables
          section below lists the relevant environment variables, and
          their defaults.

          The _e response to TeX's error prompt causes the system
          default editor to start up at the current line of the
          current file.  There is an environment variable, TEXEDIT,
          that can be used to change the editor used.  It should
          contain a string with "%s" indicating where the filename
          goes and "%d" indicating where the decimal linenumber (if
          any) goes.  For example, a TEXEDIT string for _v_i can be set
          with the _c_s_h command
              setenv TEXEDIT "/usr/ucb/vi +%d %s"

          A convenient file in the library is _n_u_l_l._t_e_x, containing
          nothing.  When TeX can't find a file it thinks you want to
          input, it keeps asking you for another file name;
          responding `null' gets you out of the loop if you don't want
          to input anything.  You can also type your EOF character
          (usually control-D).

          The _i_n_i_t_e_x and _v_i_r_t_e_x programs can be used to create fast-
          loading versions of TeX based on macro source files.  The
          _i_n_i_t_e_x program is used to create a _f_o_r_m_a_t (._f_m_t) file that
          permits fast loading of fonts and macro packages.  After
          processing the fonts and definitions desired, a \dump
          command will create the format file.  The format file is
          used by _v_i_r_t_e_x. It needs to be given a format file name as
          the first thing it reads.  A format file name is preceded by
          an &, which needs to be escaped with \ to prevent
          misinterpretation by your shell if given on the command
          line.

          Fortunately, it is no longer necessary to make explicit
          references to the format file.  The present version of TeX,
          when compiled from this distribution, looks at its own
          command line to determine what name it was called under.  It
          then uses that name, with the ``.fmt'' suffix appended, to
          search for the appropriate format file.  During
          installation,  one format file with the name _t_e_x._f_m_t, with
          only the _p_l_a_i_n._t_e_x macros defined, should have been created.
          This will be your format file when you invoke _v_i_r_t_e_x with
          the name _t_e_x.  You can also create a file _m_y_t_e_x._f_m_t using
          _i_n_i_t_e_x, so that this will be loaded when you invoke _v_i_r_t_e_x
          with the name _m_y_t_e_x.  To make the whole thing work, it is
          necessary to link _v_i_r_t_e_x to all the names of format files



     Page 3                                           (printed 3/9/91)






     TTTTEEEEXXXX((((1111))))            1111999999990000 SSSSeeeepppptttteeeemmmmbbbbeeeerrrr 22225555 ((((EEEEXXXXPPPPTTTTOOOOOOOOLLLLSSSS))))             TTTTEEEEXXXX((((1111))))



          that you have prepared.  Hard links will do for system-wide
          equivalences and Unix systems which do not use symbolic
          links.  Symbolic links can be used for access to formats for
          individual projects.  For example: _v_i_r_t_e_x can be hard linked
          to _t_e_x in the general system directory for executable
          programs, but an individual version of TeX will more likely
          be linked by a symbolic link in a privately maintained path
              ln -s $TOOLS/lib/tex/bin/virtex mytex
          in a directory such as /_h_o_m_e/_m_e/_b_i_n.

          Another approach is to set up a alias using, for example,
          the C shell:
              alias mytex virtex \&myfmt
          Besides being more cumbersome, however, this approach is not
          available to systems which do not accept aliases.  Finally,
          there is the system known as ``undump'' which takes the
          headers from an _a._o_u_t file (e.g. _v_i_r_t_e_x) and applies them to
          a core image which has been dumped by the Unix QUIT signal.
          This is very system-dependent, and produces extremely large
          files when used with a large-memory version of TeX.  This
          can produce executables which load faster, but the
          executables also consume enormous amounts of disk space.

     EEEENNNNVVVVIIIIRRRROOOONNNNMMMMEEEENNNNTTTT VVVVAAAARRRRIIIIAAAABBBBLLLLEEEESSSS
          The defaults for all environment variables are set at the
          time of compilation in a file named _s_i_t_e._h in the web2c
          distribution.  All paths are colon-separated. If you set an
          environment variable to a value that has a leading colon,
          the system default shown here is prepended.  Likewise for a
          trailing colon.  For example, if you say
              setenv TEXFONTS :/u/karl/myfonts
          TeX will search
              Furthermore, the person who installed TeX at your site
          may have chosen to allow one level of subdirectories to be
          searched automatically (by defining the symbol
          SEARCH_SUBDIRECTORIES in _s_i_t_e._h).  In that case,
          subdirectories of directories in the environment variable
          TEXFONTS_SUBDIR are also searched for fonts, and
          subdirectories of directories in the environment variable
          TEXINPUTS_SUBDIR are also searched for input files.  All the
          programs in the base TeX distribution use this same search
          method.

          Normally, TeX puts its output files in the current
          directory.  If any output file cannot be opened there, it
          tries to open it in the directory specified in the
          environment variable TEXMFOUTPUT. There is no default value
          for that variable.  For example, if you say tttteeeexxxxpaper and the
          current directory is not writable, and TEXMFOUTPUT has the
          value ////ttttmmmmpppp, TeX attempts to create ////ttttmmmmpppp////ppppaaaappppeeeerrrr....lllloooogggg (and
          ////ttttmmmmpppp////ppppaaaappppeeeerrrr....ddddvvvviiii, if any output is produced.)




     Page 4                                           (printed 3/9/91)






     TTTTEEEEXXXX((((1111))))            1111999999990000 SSSSeeeepppptttteeeemmmmbbbbeeeerrrr 22225555 ((((EEEEXXXXPPPPTTTTOOOOOOOOLLLLSSSS))))             TTTTEEEEXXXX((((1111))))



          TEXINPUTS
               Search path for \input and \openin files.  This should
               probably start with ``.''.  Default:
               ....::::$$$$TTTTOOOOOOOOLLLLSSSS////lllliiiibbbb////tttteeeexxxx////mmmmaaaaccccrrrroooossss.

          TEXINPUTS_SUBDIR
               Search path for directories with subdirectories of
               input files.  Default:
               [[[[nnnnoooonnnneeee iiiinnnn tttthhhheeee eeeexxxxppppttttoooooooollllssss TTTTeeeeXXXX iiiimmmmpppplllleeeemmmmeeeennnnttttaaaattttiiiioooonnnn]]]]

          TEXFONTS
               Search path for font metric (.tfm) files.  Default:
               ....::::$$$$TTTTOOOOOOOOLLLLSSSS////lllliiiibbbb////tttteeeexxxx////ffffoooonnnnttttssss.

          TEXFONTS_SUBDIR
               Search path for directories with subdirectories of
               fonts.  Default:
               [[[[nnnnoooonnnneeee iiiinnnn tttthhhheeee eeeexxxxppppttttoooooooollllssss TTTTeeeeXXXX iiiimmmmpppplllleeeemmmmeeeennnnttttaaaattttiiiioooonnnn]]]]

          TEXFORMATS
               Search path for format files.  Default:
               ....::::$$$$TTTTOOOOOOOOLLLLSSSS////lllliiiibbbb////tttteeeexxxx.

          TEXPOOL
               Search path for INITEX internal strings.  Default:
               ....::::$$$$TTTTOOOOOOOOLLLLSSSS////lllliiiibbbb////tttteeeexxxx.

          TEXEDIT
               Command template for switching to editor.  Default:
               """"$$$$EEEEDDDDIIIITTTTOOOORRRR ++++%%%%dddd %%%%ssss"""",,,, oooorrrr iiiiffff EEEEDDDDIIIITTTTOOOORRRR nnnnuuuullllllll oooorrrr nnnnooootttt sssseeeetttt,,,, tttthhhheeeennnn
               """"$$$$VVVVIIIISSSSUUUUAAAALLLL ++++%%%%dddd %%%%ssss"""",,,, eeeellllsssseeee iiiiffff VVVVIIIISSSSUUUUAAAALLLL nnnnuuuullllllll oooorrrr nnnnooootttt sssseeeetttt,,,, tttthhhheeeennnn
               """"vvvviiii ++++%%%%dddd %%%%ssss"""" .

     FFFFIIIILLLLEEEESSSS
          $TOOLS/lib/tex/*    TeX's library areas

          $TOOLS/lib/tex/tex.pool
                              Encoded text of TeX's messages.

          $TOOLS/lib/tex/fonts/*.tfm
                              Metric files for TeX's fonts.

          $TOOLS/lib/tex/fonts/*_n_n_n{gf,pk}
                              Bitmaps for various devices.  These
                              files are not used by TeX.

          $TOOLS/lib/tex/*.fmt
                              TeX .fmt files.

          $TOOLS/lib/tex/macros/plain.tex
                              The ``default'' macro package.




     Page 5                                           (printed 3/9/91)






     TTTTEEEEXXXX((((1111))))            1111999999990000 SSSSeeeepppptttteeeemmmmbbbbeeeerrrr 22225555 ((((EEEEXXXXPPPPTTTTOOOOOOOOLLLLSSSS))))             TTTTEEEEXXXX((((1111))))



     SSSSEEEEEEEE AAAALLLLSSSSOOOO
          latex(1), slitex(1), dvips(1), bibtex(1), detex(1)
          Donald E. Knuth, _T_h_e _T_e_X_b_o_o_k
          Leslie Lamport, _T_h_e _L_a_T_e_X _D_o_c_u_m_e_n_t _P_r_e_p_a_r_a_t_i_o_n _S_y_s_t_e_m
          Michael Spivak, _T_h_e _J_o_y _o_f _T_e_X
          _T_U_G_B_O_A_T (the publication of the TeX Users Group)

     TTTTRRRRIIIIVVVVIIIIAAAA
          TeX, pronounced properly, rhymes with ``blecchhh.''  Note
          that the proper spelling in typewriter-like media is ``TeX''
          and not ``TEX'' or ``tex.''

     AAAAUUUUTTTTHHHHOOOORRRRSSSS
          TeX was designed by Donald E. Knuth, who implemented it
          using his WEB system for Pascal programs.  It was ported to
          Unix at Stanford by Howard Trickey, and at Cornell by Pavel
          Curtis.  The version now offered with the Unix TeX
          distribution is that generated by the WEB to C system,
          written by Tomas Rokicki and Tim Morgan.




































     Page 6                                           (printed 3/9/91)



